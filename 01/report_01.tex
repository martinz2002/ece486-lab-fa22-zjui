% Lab 1: Analog Simulation
% !TEX program = lualatex
% LaTeX template for ECE 486 Lab

\documentclass[a4paper]{article}
\usepackage{indentfirst}
\usepackage{fancyhdr}
\usepackage{enumerate}
\usepackage{fontenc}
\usepackage{graphicx}
\usepackage{caption}
\usepackage{mathtools} % load AMS maths
\usepackage{amsmath}  % for math spacing
\usepackage{amssymb}  % for math spacing
\usepackage{fontspec}
\usepackage{lastpage}
\usepackage[margin=1in]{geometry} % an easy way to change page layout
                                  % Thanks to Brady Salz 
\setlength{\parskip}{1em} % 设置段落间距
% ----------------------------
%            Tikz
% ----------------------------

\usepackage{tikz}
\usetikzlibrary{shapes,arrows}
\usepackage{verbatim}

% ----------------------------

% \usepackage{kpfonts-otf}
% \usepackage{libertinus-otf}
% \usepackage{plex-otf}
\newcommand{\D}{\text{d}}

\usepackage{mathptmx}
\setmainfont{Times New Roman}

\newcommand{\score}{\hfill \underline{\hspace{0.65cm}}\,/} % for score underline
\newcommand\RR{\textsuperscript{\textregistered}~} % for registered mark
\newcommand{\EXERCISE}[1]{\subsection*{Ex \textit{#1}.}}
\newcommand{\EXERCISENAME}[2]{\subsection*{Ex #1: \textit{#2}}}
\newcommand\makeMyTitle{\maketitle \thispagestyle{firstPage}}

\pagestyle{fancy}
\fancyhf{}
\fancyhead[C]{}
\fancyhead[L]{\it Lab \LABNUMBER: \LABTITLE}
\fancyhead[R]{\it \today}
\fancyfoot[L]{\textit{Tiantian Zhong}}
\fancyfoot[C]{\thepage/\pageref{LastPage}}
\fancyfoot[R]{\textit{0000000000}}
\renewcommand{\headrulewidth}{0.5pt}
\renewcommand{\footrulewidth}{0.5pt}

\fancypagestyle{firstPage}{
    \fancyhf{}
    \fancyhead[L]{\it ECE 486 \\\it Control Systems}
    \fancyhead[C]{}
    \fancyhead[R]{\it Fall 2022 \\\it  ZJU-UIUC Institute}
    \fancyfoot[L]{}
    \fancyfoot[C]{\thepage/\pageref{LastPage}}
    \fancyfoot[R]{}
    \renewcommand{\headrulewidth}{0.5pt}
    \renewcommand{\footrulewidth}{0.5pt}
}

\newcommand{\E}{\text{e}}
% \newcommand{\Diff}{\text{d}}
\renewcommand{\Re}{\text{Re}}
\renewcommand{\Im}{\text{Im}}

\title{\textbf{Lab Report} \\ Lab \#\LABNUMBER: \textsc{\LABTITLE}}
\author{\AUTHOR \\ \ID}
\date{\today}

\def\AUTHOR{Tiantian Zhong}
\def\ID{3200110643}
% \def\REPORTTITLE{\textsc{\LABTITLE}}

\everymath{\displaystyle}

% ----------------------------
%      Tikz Configuration
% ----------------------------

\tikzstyle{block} = [draw, rectangle, 
minimum height=3em, minimum width=3em]
\tikzstyle{sum} = [draw, circle, node distance=1cm]
\tikzstyle{input} = [coordinate]
\tikzstyle{output} = [coordinate]
\tikzstyle{pinstyle} = [pin edge={to-,thin,black}]
\tikzstyle{triangle} = [isosceles triangle, draw, minimum height = 3em]
\newcommand\K{\text{k}} % kilo = 1000
\date{    \begin{tabular}{rl}
    Experiment Date: &  \EXPDATE \\
    Report Date: & \RPTDATE
\end{tabular} }


\def\LABNUMBER{1}
\def\LABTITLE{Analog Simulation}

\begin{document}

\makeMyTitle

\section*{Prelab Exercises}
\EXERCISE{a}

From Newton's 3rd law, we derive the following equation:
\begin{align*}
  ma_{tot} &= f + b\dot{x}+kx \\
  a_{tot} &= \frac{\D^2}{\D t^2} x
\end{align*}
thus we have the second-order ODE for the system,
\begin{equation*}
  m\ddot{x}=b\dot{x}+kx+f
\end{equation*}
and by simplifying the equation, we obtain
\begin{equation*}
  \ddot{x}=\frac{b}{m}\dot{x}+\frac{k}{m}x+\frac{f}{m}
\end{equation*}

Substituting the values for $m$, $b$, $k$, and $f$ we obtain
\begin{equation}
  \begin{cases}
    \ddot{x}&=\frac{0.7}{2}\dot{x}+\frac{1}{2}x+\frac{0.5}{2}\\
    x_{t=0} &= 0 \\
    \dot{x}_{t=0} &= 0.
  \end{cases}
\end{equation}

% End of Ex a.

\EXERCISE{b}

Transform the second-order ODE into a linear equatoin using Laplace Transform:
\begin{align*}
  \mathcal{L}\lbrace {\ddot{x}} \rbrace = s^2 X(s)
                    = \frac{0.7}{2} sX(s) + \frac{1}{2} X(s) + \frac{0.5}{2s}.
\end{align*}

% End of Ex b.

\EXERCISE{c}

For Figure 2(b) in the manual we can derive a group of equations based on KVL,
\begin{align}
   q_C &= Ce_o(t) \\
  e_i(t) - e_o(t) &= i(t)R \\
  \D q_C &= i(t)\D t
\end{align}
which gives
\begin{align*}
  &e_i(t) - \frac{q_C}{C} = i(t) R
\end{align*}
i.e.
\begin{align*}
  &e_i(t) - \frac{1}{C}\int i(t)\D t- i(t)R = 0
\end{align*}

\section*{Plots}

\section*{Data Analysis}

\end{document}
